\label{sec:certificates} 
\chapter{Certificates}

The ER3 uses \href{http://en.wikipedia.org/wiki/X.509}{X.509 certificates}.
There are two types of certificates, the developer certificate\supref{sec:developer certificate} for signing packages\supref{sec:signed packages} and the installation certificate\supref{sec:installation certificate} for licensing\supref{sec:licensing}.

\section[sec:developer certificate]{Developer certificate}

Every registered person or company can only have one developer certificate.
The developer certificate is used to sign packages\supref{sec:signed packages}.
The ER3 Manager\supref{sec:er3 manager} can use the signature to validate the integrity and origin.

\section[sec:installation certificate]{Installation certificate}

An installation certificate is only required to use the extension store\supref{sec:extension store}.
It is generated through the ER3 Manager\supref{sec:er3 manager}.
The public key is stored in the Repository Portal\supref{sec:repository portal}, associated to a registered person or company and the installation domain (just for human identification).
A single person or company can register multiple installation certificates.

\paragraph{}
An installation certificate should be unique, this means every installation get its own certificate.
While generating a new installation certificate the ER3 Manager\supref{sec:er3 manager} stored some unique informations with the certificate, the installation domain and the installation path.
If these informations change, the ER3 Manager\supref{sec:er3 manager} ask the user, if a new certificate should be generated.
Normally the answer is no, because in most cases these informations change while put an installation from development to productive, change the provider or something similar.
But if an installation is duplicated to create a new website, for example by using a copy master. This question can prevent installation certificate duplication.
In most cases, this should not be necessary, because an installation certificate is first generated if it is needed. Or in other words, the first time the extension store\supref{sec:extension store} is opened from the ER3 Manager\supref{sec:er3 manager}.

\section[sec:key exchange]{Key exchange}

The public keys are stored in the Repository Portal\supref{sec:repository portal}.
A public key can only be stored by registered persons and companies, this means it is not an open public key storage.
The portal also provide a \href{http://en.wikipedia.org/wiki/Revocation_list}{certification revocation list} for deleted keys.
A key get invalid if the owning person or company account is deleted.

\section[sec:certificate dependencies]{Dependencies}

The php openssl module is required for this functionality.
This module should be available since php 4.

\section[sec:signed packages]{Signed packages}

A developer can sign his packages.
Signed packages provide more security because the user can validate the origin of the package.
All developers should sign there packages, the extension builder\supref{sec:extension builder} normally do this for the developer.
If the ER3 Manager\supref{sec:er3 manager} try to install an unsigned packages it gives a security warning to the user.
