\section[sec:repository portal]{Repository Portal}

The portal is a public interface. It links external services to provide additional informations for extensions, that are available on the world wide web.

First it is a repository browser. Users can browse and search extensions and show details of them.
Secondary it is a information aggregator to display related informations.

A package contains links to other sources, see \ref{sec:package information file}.
\lstset{language=XML} 
\begin{lstlisting}[caption=XML example: links to other sources]
...
<ctx:link rel="homepage" href="http://www.contao.org">
    <ctx:description>Official project homepage.</ctx:description>
</ctx:link>
<ctx:link rel="manual" href="http://www.contao.org/documentation.html" language="en">
    <ctx:description>Official User-Manual.</ctx:description>
</ctx:link>
<ctx:link rel="help" href="http://en.contaowiki.org" language="en">
    <ctx:description>Inofficial Wiki.</ctx:description>
</ctx:link>
<ctx:link rel="video" href="http://www.youtube.com/watch?feature=player_embedded&amp;v=8HkQVmfjjYc">
    <ctx:description>Introduction into TypoLIGHT/Contao.</ctx:description>
</ctx:link>
<ctx:link rel="forum" href="http://www.contao-community.de" language="de">
    <ctx:description>German contao community.</ctx:description>
</ctx:link>
<ctx:link rel="tracker" href="http://dev.contao.org">
    <ctx:description>Official bug and issue tracker.</ctx:description>
</ctx:link>
<ctx:link rel="scm" href="http://svn.contao.org">
    <ctx:description>Official svn repository.</ctx:description>
</ctx:link>
...
\end{lstlisting}
The portal use this links and try to fetch related informations from these systems. I.e. this can be bugs and issues from a tracking system like Redmine or Github. Or screencast from YouTube. Or tutorials from a MediaWiki.

The Portal is not only an information aggregator. Not only the developer can link external informations to a extensions.
Other users can link ther own informations, like tutorials, screencasts or similar to an extension.
Theses links are stored in the portal, but not in the extension information. This is the social networking character of the Portal.

\subsection[sec:information kinds and system connectors]{Information kinds and system connectors}

The Portal use multiple connectors to get informations from other systems.
First we have to define the kind of informations, we are able to aggregate.

\begin{itemize}
\item Bugs and issues\\
The portal can display bugs and issues to show the current development status and upstream development.
\item Manuals and Tutorials - Wiki informations\\
One of the best ways to manage manuals and tutorials is to use a wiki. The portal can display related wiki articles.
\item Videos and Screencasts\\
Videos and screencasts are another kind of tutorials. The portal can display videos hosted on video streaming platforms like YouTube.
\item Source code\\
If an scm repository is configured, the portal can directly link to the source code.
\end{itemize}

These are just a few possible information kinds, the repository may use.

Next we have to define how the informations are aggregated.
A lot of systems like \hyperlink{http://www.mediawiki.org/wiki/API:Main_page/de}{MediaWiki}, \hyperlink{http://code.google.com/intl/de-DE/apis/youtube/overview.html}{YouTube} and \hyperlink{http://developer.github.com/v3/}{github} provide external APIs.
The Portal use connectors to this systems to get informations.
Because not all theses services are uniq, like MediaWiki the systems have to be configures dynamicaly.
All users are allowed to request adding a service, but only administrators are able to add them finaly into the portal.
Initialy there may be these connectors availeable:

\small
\begin{longtable}{|p{0.25\textwidth}|c|c|c|c}
\caption{Information system connectors and information kinds} \\
\hline
\label{tab:information system connectors and information kinds}
\textbf{System} & \textbf{Tracking} & \textbf{Wiki} & \textbf{Video} & \textbf{SCM}
\\ \hline
 MediaWiki & & X & &
\\ \hline
 Redmine & X & X & &
\\ \hline
 Github & X & X & & X
\\ \hline
 YouTube & & & X &
\\ \hline
 Vimeo & & & X &
\\ \hline
\end{longtable}
\normalsize
